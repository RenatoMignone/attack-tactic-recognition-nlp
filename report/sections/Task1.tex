\section{Dataset Characterization}
In this section, we explore the provided training and test datasets.

\paragraph{Labels exploration} \Cref{fig:task1_ecdf_bash_words} illustrates the distribution of tags in both the training and test sets. The datasets comprise seven distinct tags: \emph{Defense Evasion}, \emph{Discovery}, \emph{Execution}, \emph{Impact}, \emph{Not Malicious Yet}, \emph{Other}, and \emph{Persistence}. Generally, the number of Bash words assigned to a tag is higher in the training set than in the test set. In fact, the training set contains 11475 Bash words distributed among 251 sessions, while the test set only 6197 (108 sessions). Both sets exhibit a similar distribution profile: \emph{Discovery} ($\approx\SI{50}{\percent}$) is the most prevalent tag, followed by \emph{Execution} ($\approx\SI{25}{\percent}$), and \emph{Persistence} ($\approx\SI{10}{\percent}$). The remaining four tags are significantly less frequent, each appearing in less than \SI{4}{\percent} instances.

\paragraph{The \texttt{echo} command} The \texttt{echo} command has been assigned to 6 different tags: \emph{Persistence} (104 times), \emph{Execution} (39 times), \emph{Discovery} (31 times), \emph{Not Malicious Yet} (8 times), \emph{Impact} (6 times), and \emph{Other} (4 times). We will now analyze two example of sessions: one in which \texttt{echo} is labelled as \emph{Persistence} and one as \emph{Execution}.

\textbf{(\texttt{echo}, Persistence):} \texttt{[\dots]; echo ``root:HGbB4i9gUXMh' | chpasswd | bash ; [\dots]}. The \texttt{echo} outputs the string \texttt{root:HGbB4i9gUXMh} (username:password hash), which is piped to \texttt{chpasswd}, that updates user passwords from standard input. By changing the \texttt{root} user's password to a known value, the attacker gains persistent administrative access to the compromised system.

\textbf{(\texttt{echo}, Execution):} \texttt{[\dots]; echo
``base64\_payload'' | base64 --decode | bash ;}. The \texttt{echo} outputs a base64 payload, that is then decoded and executed. Hence, the command immediately executes arbitrary code, probably a malware binary.

In both cases, the \texttt{echo} command is used to output a string. However, the meaning of this string and the goal of the subsequent commands influence the labelling of \texttt{echo} itself: \emph{Persistence} in the first case, and \emph{Execution} in the second one.

\paragraph{Bash words exploration} \Cref{fig:task1_ecdf_bash_words} shows the ECDF of Bash words per session in both the training and test sets. The distribution shows that most sessions are relatively short, with \SI{50}{\percent} containing less than 25 words, and $\approx\SI{80}{\percent}$ less than 100 words. While both datasets share the same minimum (2 words) and maximum (224 words) lengths, the test set is generally characterized by longer sessions. Specifically, the mean session length in the test set is \SI{57.38}{\words}, compared to \SI{45.70}{\words} in the training set. This right-skewed distribution indicates the presence of a few longer sessions that contrast with the high frequency of shorter ones.

\begin{figure}
	\centering
    \subfloat[][{Tag distribution.}\label{fig:task1_tags_distribution}]
	{\includegraphics[width=.54\linewidth]{img/Task1/task1_tags_distribution.png}} \quad
	\subfloat[][{ECDF of Bash words per session.}\label{fig:task1_ecdf_bash_words}]
	{\includegraphics[width=.429\linewidth]{img/Task1/task1_ecdf_length.png}} \\
  	\caption{Plots about Dataset Characterization. Both plots refer to the training and test sets.}{}\label{fig:task1}
\end{figure}